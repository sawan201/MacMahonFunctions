\documentclass[reqno]{article}
\usepackage[dvipsnames]{xcolor}
\usepackage[color]{xypic}
\usepackage{amsmath,amsthm,amssymb,mathrsfs,graphics,enumitem,ytableau,tcolorbox,diagbox,fullpage,float,blkarray,etoolbox,thmtools,palatino,bbm}
\usepackage{tikz,tikz-cd}
\usepackage[backref=page]{hyperref}
\usepackage{soul}
\usepackage{cprotect} % to allow verbatim text in footnotes
\raggedbottom
\parskip=10bp
\parindent=0bp
\setlist[itemize]{noitemsep, topsep=0pt}
\setlist[enumerate]{noitemsep, topsep=0pt}

% tikz stuff
\usetikzlibrary{calc,3d,patterns}
\tikzset{>=latex}
\colorlet{cyan}[rgb]{cyan}
\definecolor{verylight}{gray}{.85}
\definecolor{light}{gray}{.75}
\definecolor{lightish}{gray}{.6}
\definecolor{medium}{gray}{.5}
\definecolor{dark}{gray}{.25}
\definecolor{kindofdark}{gray}{.4}
\definecolor{posetshade}{gray}{.85} %% standard shading color for all poset diagrams

% colors
\newcommand{\Red}[1]{{\color{red}{#1}}}
\newcommand{\RED}[1]{{\color{red}{\boldmath\textbf{#1}\unboldmath}}}
\newcommand{\Blue}[1]{{\color{blue}{#1}}}
\newcommand{\BLUE}[1]{{\color{blue}{\boldmath\textbf{#1}\unboldmath}}}
\newcommand{\Green}[1]{{\color{ForestGreen}{#1}}}
\newcommand{\GREEN}[1]{{\color{ForestGreen}{\boldmath\textbf{#1}\unboldmath}}}

% hyperlinks
\hypersetup{colorlinks, citecolor=red, filecolor=black, linkcolor=blue, urlcolor=blue}
\newcommand{\hreftext}[2]{\href{#1}{\Blue{#2}}}
\newcommand{\hrefurl}[2]{\href{#1}{\Blue{\tt #2}}}

% spacing
\newcommand{\bigpad}{\rule[-14mm]{0mm}{30mm}}
\newcommand{\smallpad}{\rule[-1.5mm]{0mm}{5mm}}
\newcommand{\bigsmallpad}{\rule[-2mm]{0mm}{6mm}}
\newcommand{\Pad}{\rule[-5mm]{0mm}{12mm}}
\newcommand{\pad}{\rule[-3mm]{0mm}{8mm}}
\newcommand{\padup}{\rule{0mm}{5mm}}
\newcommand{\paddown}{\rule[-3mm]{0mm}{2mm}}
\newcommand{\bigpaddown}{\rule[-8mm]{0mm}{5mm}}
\newcommand{\smallpaddown}{\rule[-1.5mm]{0mm}{0mm}}
\newcommand{\blank}{\rule{1.25in}{0.25mm}}
\newcommand{\excise}[1]{}
\newcommand{\commentout}[1]{}
\newcommand{\yell}[1]{\fbox{\rule[-1mm]{0mm}{4mm} \large\bf #1 }}
\newcommand{\bang}{$\bullet$\quad}
\newcommand{\indnt}{\phantom{.}\qquad}
\newcommand{\littleline}{\begin{center}\rule{4in}{0.5bp}\end{center}}

% including PDF figures
\newcommand{\includefigure}[3]{{
  \begin{center}
  \resizebox{#1}{#2}{\includegraphics{{#3}}}
  \end{center}}}
\newcommand{\includefigurewithinmath}[3]{{
  \resizebox{#1}{#2}{\includegraphics{{#3}}}}}
\newcommand{\insertfigure}[3]{\resizebox{#1}{#2}{\includegraphics{{#3}}}}

\newcommand{\defterm}[1]{\boldmath\textbf{#1}\unboldmath}
\newcommand{\embolden}[1]{\boldmath\textbf{#1}\unboldmath}

% special combinatorial objects get put in sans-serif

\newcommand{\Bool}{\mathsf{Bool}}	% boolean lattice
\newcommand{\Braid}{\mathsf{Br}}	% braid arrangement
\newcommand{\CST}{\mathsf{CST}}		% column-strict tableaux
\newcommand{\GP}{\mathsf{GP}}		% generalized permutations
\newcommand{\NC}{\mathsf{NC}}		% Noncrossing partitions
\newcommand{\nc}{\mathsf{nc}}		% ... and their cardinalities
\newcommand{\Par}{\mathsf{Par}}		% partitions
\newcommand{\RSK}{\mathsf{RSK}}		% RSK algorithm
\newcommand{\Shi}{\mathsf{Shi}}		% Shi arrangement
%\newcommand{\Sp}{\mathsf{Sp}}		% Specht module
\newcommand{\SYT}{\mathsf{SYT}}		% standard Young tableaux

% math operators galore

\DeclareMathOperator{\aff}{aff}
\DeclareMathOperator{\Ch}{Ch}
\DeclareMathOperator{\Char}{char}
\DeclareMathOperator{\ch}{\mathbf{ch}}
\DeclareMathOperator{\colspace}{colspace}
\DeclareMathOperator{\colsum}{colsum}
\DeclareMathOperator{\Comp}{Comp}
\DeclareMathOperator{\conv}{conv}
\DeclareMathOperator{\corank}{corank}
\DeclareMathOperator{\deln}{del}
\DeclareMathOperator{\diag}{diag}
\DeclareMathOperator{\ess}{ess}
\DeclareMathOperator{\flip}{flip}
\DeclareMathOperator{\Gr}{\mathsf{Gr}} % Grassmannian
\DeclareMathOperator{\Hilb}{Hilb}
\DeclareMathOperator{\Hom}{Hom}
\DeclareMathOperator{\id}{id}
\DeclareMathOperator{\Id}{Id}
\DeclareMathOperator{\im}{im}
\DeclareMathOperator{\indeg}{indeg}
\DeclareMathOperator{\inv}{inv}
\DeclareMathOperator{\Irr}{Irr}
\DeclareMathOperator{\Ind}{Ind}
\DeclareMathOperator{\Int}{Int}
\DeclareMathOperator{\inte}{int}  %% since \int is an integral sign
\DeclareMathOperator{\lcm}{lcm}
\DeclareMathOperator{\link}{link}
\DeclareMathOperator{\Mob}{M\ddot{o}b}
\DeclareMathOperator{\nullity}{nullity}
\DeclareMathOperator{\nullspace}{nullspace}
\DeclareMathOperator{\Poin}{Poin}
\DeclareMathOperator{\proj}{proj}
\DeclareMathOperator{\rank}{rank}
\DeclareMathOperator{\relint}{relint}
\DeclareMathOperator{\Res}{Res}
\DeclareMathOperator{\row}{row}
\DeclareMathOperator{\rowspace}{rowspace}
\DeclareMathOperator{\rowsum}{rowsum}
\DeclareMathOperator{\Sd}{Sd}
\DeclareMathOperator{\sign}{sign}
\DeclareMathOperator{\Span}{span}
\DeclareMathOperator{\Star}{star}
\DeclareMathOperator{\supp}{supp}
\DeclareMathOperator{\sh}{sh}
\DeclareMathOperator{\tr}{tr}
\DeclareMathOperator{\Vol}{Vol}
\DeclareMathOperator{\vol}{vol}
\DeclareMathOperator{\width}{width}
\DeclareMathOperator{\wt}{wt}

%theorem declarations using thmtools
\declaretheorem[style=plain]{theorem}
\declaretheorem[style=plain,sibling=theorem]{lemma}
\declaretheorem[style=plain,sibling=theorem]{corollary}
\declaretheorem[style=plain,sibling=theorem]{conjecture}
\declaretheorem[style=plain,sibling=theorem]{proposition}
\declaretheorem[style=definition,sibling=theorem]{definition}
\declaretheorem[style=definition,qed=$\blacktriangleleft$,sibling=theorem]{example}
\declaretheorem[style=definition,sibling=theorem]{remark}
\declaretheorem[style=definition,sibling=theorem]{note}
\declaretheorem[style=definition]{problem}

% many more useful symbols
\newcommand{\0}{\emptyset}
\newcommand{\Alt}{\mathfrak{A}}
\newcommand{\bd}{\partial}
\newcommand{\bs}{\backslash}
%\newcommand{\CHI}{\raisebox{0.065cm}{$\chi$}}
\newcommand{\CHI}{\chi^{\phantom{*}}} %% only to be used with subscripts
\newcommand{\Cl}{C\ell}
\newcommand{\compn}{\models} % for compositions
\newcommand{\covers}{\gtrdot}
\newcommand{\coveredby}{\lessdot}
\newcommand{\dedge}[1]{\overrightarrow{{#1}}}

% domination order and negations:
\newcommand{\dom}{\rhd}         \newcommand{\ndom}{\ntriangleright}
\newcommand{\domeq}{\unrhd}     \newcommand{\ndomeq}{\ntrianglerighteq}
\newcommand{\domby}{\lhd}       \newcommand{\ndomby}{\ntriangleleft}
\newcommand{\dombyeq}{\unlhd}   \newcommand{\ndombyeq}{\ntrianglelefteq}

\newcommand{\dprod}{\displaystyle\prod}
\newcommand{\dsum}{\displaystyle\sum}
\newcommand{\ep}{\varepsilon}
\newcommand{\eqdef}{\overset{{\rm def}}{\equiv}}
\newcommand{\fld}{\Bbbk}
\newcommand{\half}{\frac12}
\newcommand{\hatzero}{\hat{\mathbf{0}}}
\newcommand{\hatone}{\hat{\mathbf{1}}}
\newcommand{\inj}{\hookrightarrow} % injection arrow
\newcommand{\isom}{\cong}
\newcommand{\join}{\vee}
\renewcommand{\Join}{\bigvee}
\newcommand{\Laff}{L^{\text{aff}}}
\newcommand{\lin}[1]{\overleftrightarrow{{#1}}}
\newcommand{\meet}{\wedge}
\newcommand{\Meet}{\bigwedge}
\newcommand{\ov}[1]{\overline{{#1}}}
\newcommand{\partn}{\vdash}
\newcommand{\qqandqq}{\qquad\text{and}\qquad}
\newcommand{\qandq}{\quad\text{and}\quad}
\newcommand{\qand}{\quad\text{and}}
\newcommand{\qqorqq}{\qquad\text{or}\qquad}
\newcommand{\qorq}{\quad\text{or}\quad}
\newcommand{\qor}{\quad\text{or}}
\newcommand{\qbin}[2]{\genfrac[]{0pt}{}{#1}{#2}_q} % q-binomial coefficients
\newcommand{\sd}{\triangle} % symmetric difference
\newcommand{\simK}{\underset{K}{\sim}} % Knuth equivalence
\newcommand{\simJ}{\underset{J}{\sim}} % jeu de taquin equivalence
\newcommand{\sm}{\setminus}
\renewcommand{\st}{\colon}  %%  \st means strikeout in soul package
\newcommand{\surj}{\twoheadrightarrow} % surjection arrow
\newcommand{\Sym}{\mathfrak{S}} % symmetric group
\newcommand{\un}[1]{\underset{*}{#1}}
\newcommand{\unA}{\un{A}}
\newcommand{\unB}{\un{B}}
\newcommand{\unw}{\un{w}}
\newcommand{\unx}{\un{x}}
\newcommand{\uny}{\un{y}}
\newcommand{\unz}{\un{z}}
\newcommand{\x}{\times}
\newcommand{\zerovec}{\vec{\mathbf{0}}}

% sans-serif letters
\newcommand{\sfs}{\mathsf{S}}
\newcommand{\sfu}{\mathsf{U}}
\newcommand{\sfw}{\mathsf{W}}
\newcommand{\sfx}{\mathsf{X}}
\newcommand{\sfy}{\mathsf{Y}}
\newcommand{\sfz}{\mathsf{Z}}

% boldface letters
\renewcommand{\aa}{\mathbf{a}}
\newcommand{\bb}{\mathbf{b}}
\newcommand{\cc}{\mathbf{c}}
\newcommand{\dd}{\mathbf{d}}
\newcommand{\ee}{\mathbf{e}}
\newcommand{\ff}{\mathbf{f}}
\newcommand{\kk}{\mathbf{k}}
\newcommand{\bl}{\mathbf{l}}
\newcommand{\mm}{\mathbf{m}}
\newcommand{\nn}{\mathbf{n}}
\newcommand{\pp}{\mathbf{p}}
\newcommand{\qq}{\mathbf{q}}
\newcommand{\uu}{\mathbf{u}}
\newcommand{\vv}{\mathbf{v}}
\newcommand{\ww}{\mathbf{w}}
\newcommand{\xx}{\mathbf{x}}
\newcommand{\yy}{\mathbf{y}}
\newcommand{\zz}{\mathbf{z}}

% calligraphic letters
\newcommand{\A}{\mathcal{A}}
\newcommand{\B}{\mathcal{B}}
\newcommand{\C}{\mathcal{C}}
\newcommand{\D}{\mathcal{D}}
\newcommand{\EE}{\mathcal{E}}
\newcommand{\G}{\mathcal{G}}
\newcommand{\HH}{\mathcal{H}}
\newcommand{\M}{\mathcal{M}}
\newcommand{\NN}{\mathcal{N}}
\renewcommand{\O}{\mathcal{O}}
\renewcommand{\P}{\mathcal{P}}
\newcommand{\T}{\mathcal{T}}

% mathsf letters (for Hopf monoids)
\newcommand{\HHH}{\mathsf{H}}
\newcommand{\hhh}{\mathsf{h}}
\newcommand{\LLL}{\mathsf{L}}
\newcommand{\el}{\pmb{\ell}}
\newcommand{\MMM}{\mathsf{M}}
\newcommand{\mmm}{\mathsf{m}}
\newcommand{\antipode}{\mathsf{S}}

%% fancy script letters -- requires mathrsfs package
\newcommand{\BB}{\mathscr{B}}
\newcommand{\CC}{\mathscr{C}}
\newcommand{\FF}{\mathscr{F}}
\newcommand{\II}{\mathscr{I}}
\newcommand{\LL}{\mathscr{L}}
\newcommand{\PP}{\mathscr{P}}
\renewcommand{\SS}{\mathscr{S}}
\newcommand{\XX}{\mathscr{X}}

%% blackboard-bold letters for sets of numbers
\newcommand{\Aa}{\mathbb{A}}
\newcommand{\Cc}{\mathbb{C}}
\newcommand{\Ff}{\mathbb{F}}
\newcommand{\Nn}{\mathbb{N}}
\newcommand{\Pp}{\mathbb{P}}
\newcommand{\Qq}{\mathbb{Q}}
\newcommand{\Rr}{\mathbb{R}}
\newcommand{\Ss}{\mathbb{S}}
\newcommand{\Zz}{\mathbb{Z}}
\newcommand{\One}{\mathbb{1}} % all-ones vector

%% representations and characters
\newcommand{\rhodef}{\rho^{\phantom{*}}_{{\rm def}}}
\newcommand{\rhotriv}{\rho^{\phantom{*}}_{{\rm triv}}}
\newcommand{\rhosign}{\rho^{\phantom{*}}_{{\rm sign}}}
\newcommand{\rhoreg}{\rho^{\phantom{*}}_{{\rm reg}}}
\newcommand{\rhostd}{\rho^{\phantom{*}}_{{\rm std}}}
\newcommand{\rhogeo}{\rho^{\phantom{*}}_{{\rm geo}}}
\newcommand{\chidef}{\chi^{\phantom{*}}_{{\rm def}}}
\newcommand{\chitriv}{\chi^{\phantom{*}}_{{\rm triv}}}
\newcommand{\chisign}{\chi^{\phantom{*}}_{{\rm sign}}}
\newcommand{\chistd}{\chi^{\phantom{*}}_{{\rm std}}}
\newcommand{\chigeo}{\chi^{\phantom{*}}_{{\rm geo}}}
\newcommand{\chitrivSn}{\chi^n_{{\rm triv}}}
\newcommand{\chisignSn}{\chi^n_{{\rm sign}}}
\newcommand{\chireg}{\chi^{\phantom{*}}_{{\rm reg}}}
\newcommand{\chiother}{\chi^{\phantom{*}}_{{\rm other}}}
\newcommand{\scp}[2]{\left\langle #1,\:#2\right\rangle}
\newcommand{\scpG}[2]{\left\langle #1,\:#2\right\rangle_G}
\newcommand{\scpH}[2]{\left\langle #1,\:#2\right\rangle_H}

%% shorthands for lines in Hasse diagrams using xypic
\newcommand{\Lin}[1]{\ar@{-}[#1]}

\newcommand{\linel}{\ar@{-}[l]}
\newcommand{\linell}{\ar@{-}[ll]}
\newcommand{\liner}{\ar@{-}[r]}
\newcommand{\linerr}{\ar@{-}[rr]}
\newcommand{\lined}{\ar@{-}[d]}
\newcommand{\linedr}{\ar@{-}[dr]}
\newcommand{\linedl}{\ar@{-}[dl]}
\newcommand{\linedd}{\ar@{-}[dd]}
\newcommand{\lineddr}{\ar@{-}[ddr]}
\newcommand{\linedrr}{\ar@{-}[drr]}
\newcommand{\linedrrr}{\ar@{-}[drrr]}
\newcommand{\lineddl}{\ar@{-}[ddl]}
\newcommand{\linedll}{\ar@{-}[dll]}
\newcommand{\linedlll}{\ar@{-}[dlll]}
\newcommand{\linedllll}{\ar@{-}[dllll]}
\newcommand{\lineu}{\ar@{-}[u]}
\newcommand{\lineur}{\ar@{-}[ur]}
\newcommand{\lineul}{\ar@{-}[ul]}
\newcommand{\lineuu}{\ar@{-}[uu]}
\newcommand{\lineuur}{\ar@{-}[uur]}
\newcommand{\lineurr}{\ar@{-}[urr]}
\newcommand{\lineurrr}{\ar@{-}[urrr]}
\newcommand{\lineuul}{\ar@{-}[uul]}
\newcommand{\lineull}{\ar@{-}[ull]}
\newcommand{\lineulll}{\ar@{-}[ulll]}

%% shorthands for lines in Hasse diagrams using tikz-cd
\newcommand{\tlinel}{\arrow[l,dash]}
\newcommand{\tlinell}{\arrow[ll,dash]}
\newcommand{\tliner}{\arrow[r,dash]}
\newcommand{\tlinerr}{\arrow[rr,dash]}
\newcommand{\tlined}{\arrow[d,dash]}
\newcommand{\tlinedr}{\arrow[dr,dash]}
\newcommand{\tlinedl}{\arrow[dl,dash]}
\newcommand{\tlinedd}{\arrow[dd,dash]}
\newcommand{\tlineddr}{\arrow[ddr,dash]}
\newcommand{\tlinedrr}{\arrow[drr,dash]}
\newcommand{\tlinedrrr}{\arrow[drrr,dash]}
\newcommand{\tlineddl}{\arrow[ddl,dash]}
\newcommand{\tlinedll}{\arrow[dll,dash]}
\newcommand{\tlinedlll}{\arrow[dlll,dash]}
\newcommand{\tlinedllll}{\arrow[dllll,dash]}
\newcommand{\tlineu}{\arrow[u,dash]}
\newcommand{\tlineur}{\arrow[ur,dash]}
\newcommand{\tlineul}{\arrow[ul,dash]}
\newcommand{\tlineuu}{\arrow[uu,dash]}
\newcommand{\tlineuur}{\arrow[uur,dash]}
\newcommand{\tlineurr}{\arrow[urr,dash]}
\newcommand{\tlineurrr}{\arrow[urrr,dash]}
\newcommand{\tlineuul}{\arrow[uul,dash]}
\newcommand{\tlineull}{\arrow[ull,dash]}
\newcommand{\tlineulll}{\arrow[ulll,dash]}

\renewcommand*\circled[1]{\tikz[baseline=(char.base)]{\node[shape=circle,draw,inner sep=2pt] (char) {#1};}}

% all the following code is required to get a dot in a cup for the disjoint-union symbol!
\makeatletter
\def\moverlay{\mathpalette\mov@rlay}
\def\mov@rlay#1#2{\leavevmode\vtop{%
   \baselineskip\z@skip \lineskiplimit-\maxdimen
   \ialign{\hfil$\m@th#1##$\hfil\cr#2\crcr}}}
\newcommand{\charfusion}[3][\mathord]{
    #1{\ifx#1\mathop\vphantom{#2}\fi
        \mathpalette\mov@rlay{#2\cr#3}
      }
    \ifx#1\mathop\expandafter\displaylimits\fi}
\makeatother

\newcommand{\cupdot}{\charfusion[\mathbin]{\cup}{\cdot}}
\newcommand{\bigcupdot}{\charfusion[\mathop]{\bigcup}{\cdot}}
\newcommand{\dju}{\cupdot}


\newcommand{\Mac}{\mathfrak{M}}
\DeclareMathOperator{\type}{type}
\begin{document}
\section{Expanding a power-sum MacMahon symmetric function in the monomial basis}
\bigskip

Fix a positive integer $r$.  Throughout, we will work with MacMahon symmetric functions in $r$ alphabets.  The complete set of variables is
\[\{x_{i,j}\mid i\in\Pp,\ j\in[r]\}.\]

Fix vector partitions $\lambda=(\lambda_1,\dots,\lambda_\ell)$ and $\mu=(\mu_1,\dots,\mu_m)$ of the same vector, where
\begin{align*}
\lambda_i&=(\lambda_{i,1},\dots,\lambda_{i,r})\in\Pp^r\setminus\{\vec0\},\\
\mu_i&=(\mu_{i,1},\dots,\mu_{i,r})\in\Pp^r\setminus\{\vec0\}.
\end{align*}

The monomial MacMahon symmetric function indexed by $\mu$ is the sum of all monomials in the diagonal $\mathfrak{S}_n$-orbit of the monomial
\[x_\mu = \prod_{i=1}^m \prod_{j=1}^r x_{i,j}^{\mu_{i,j}}.\]

The power-sum MacMahon symmetric function indexed by $\lambda$ is
\begin{align*}
P_\lambda
&= \prod_{i=1}^\ell P_{\lambda_i}\\
&= \prod_{i=1}^\ell \sum_{k\in\Pp} x_{k,1}^{\lambda_{i,1}}\cdots x_{k,r}^{\lambda_{i,r}}\\
&= \sum_{(k_1,\dots,k_\ell)\in\Pp^\ell} \prod_{i=1}^\ell x_{k_i,1}^{\lambda_{i,1}}\cdots x_{k_i,r}^{\lambda_{i,r}}.
\end{align*}

Thus the coefficient of $x_\mu$ in $P_\lambda$ is
\[\#\left\{(k_1,\dots,k_\ell)\in\Pp^\ell \mid \sum_{i\colon\,k_i=h}\lambda_i=\mu_h\ \ \forall h\in[m]\right\}.\]
Note that we can replace $\Pp^\ell$ by $m^\ell$.

For example, fix $r=2$.  Let $\mu=(1,1)(1,2)$, so that
\[x_\mu = x_{1,1}^1 x_{1,2}^1 x_{2,1}^1 x_{2,2}^2\]
and let $\lambda=(1,1)(1,0)(0,1)(0,1)$ so that $\ell=4$ and
\begin{align*}
P_\lambda
&= \sum_{(k_1,\dots,k_4)} \prod_{i=1}^4 x_{k_i,1}^{\lambda_{i,1}} x_{k_i,2}^{\lambda_{i,2}}\\
&= \sum_{(k_1,\dots,k_4)} 
x_{k_1,1}^{\lambda_{1,1}} x_{k_1,2}^{\lambda_{1,2}}
x_{k_2,2}^{\lambda_{2,1}} x_{k_2,2}^{\lambda_{2,2}}
x_{k_3,1}^{\lambda_{3,1}} x_{k_3,2}^{\lambda_{3,2}}
x_{k_4,1}^{\lambda_{4,1}} x_{k_4,2}^{\lambda_{4,2}}\\
&= \sum_{(k_1,\dots,k_4)} 
x_{k_1,1}^1 x_{k_1,2}^1
x_{k_2,2}^1 x_{k_2,2}^0
x_{k_3,1}^0 x_{k_3,2}^1
x_{k_4,1}^0 x_{k_4,2}^1
\end{align*}
so in order to make this monomial equal to $x_\mu$, we need
\[\sum_{i:\ k_i=1} \lambda_i = \mu_1 = (1,1),
\qquad\sum_{i:\ k_i=2} \lambda_i = \mu_2 = (1,2).\]
The solutions for $(k_1,k_2,k_3,k_4)$ and the corresponding expressions for $x_\mu$ are
\begin{align*}
(1,2,2,2) & & x_{1,1}^1 x_{1,2}^1 x_{2,1}^1 x_{2,2}^2 &= (x_{1,1}^1 x_{1,2}^1)(x_{2,2}^1 x_{2,2}^0)(x_{2,1}^0 x_{2,2}^1)(x_{2,1}^0 x_{2,2}^1)\\
(2,1,1,2) & & x_{1,1}^1 x_{1,2}^1 x_{2,1}^1 x_{2,2}^2 &= (x_{2,1}^1 x_{2,2}^1)(x_{1,2}^1 x_{1,2}^0)(x_{1,1}^0 x_{1,2}^1)(x_{2,1}^0 x_{2,2}^1)\\
(2,1,2,1) & & x_{1,1}^1 x_{1,2}^1 x_{2,1}^1 x_{2,2}^2 &= (x_{2,1}^1 x_{2,2}^1)(x_{1,2}^1 x_{1,2}^0)(x_{2,1}^0 x_{2,2}^1)(x_{1,1}^0 x_{1,2}^1)\\
\end{align*}
So the coefficient is 3.  This could certainly be calculated algorithmically.

\section{Transitions between bases, from Rosas 2001}

First, a standard linear algebra fact.
Let $V$ be a vector space of finite dimension~$n$ over ground field $\Bbbk$ (it doesn't matter what it is).

\begin{proposition} \label{expansion}
Suppose that $B=\{v_1,\dots,v_n\}$ and $B^*=\{v_1^*,\dots,v_n^*\}$ are bases for $V$ that are orthogonal, i.e.,
\[\langle v_i,v^*_j\rangle=\delta_{ij} k_i\]
for some nonzero constants $k_1,\dots,k_n$.  Then the unique expansion of $u$ with respect to $B$ is
\begin{equation} \label{expand-in-basis}
u = \sum_{i=1}^n \frac{\langle v_i^*,\,u\rangle}{k_i} v_i.
\end{equation}
\end{proposition}

\begin{proof}
For every $j\in[n]$ we have
\[
\left\langle v_j^*,\ \sum_{i=1}^n \frac{\langle v_i^*,\,u\rangle}{k_i} v_i\right\rangle
= \sum_{i=1}^n \frac{\langle v_i^*,\,u\rangle}{k_i} \langle v_j^*,\,v_i\rangle
= \sum_{i=1}^n \langle v_i^*,\,u\rangle \delta_{ij}
= \langle v_j^*,\,u\rangle.\]
That is, every element of $B^*$ has the same inner product with the left and right-hand sides of~\eqref{expand-in-basis}.
This is necessary and sufficient for equality.
\end{proof}

This proposition reduces the problem of expanding a vector in a particular basis $B$ to computing inner products with an orthogonal basis $B^*$.
(The best possible case is that $k_i=1$ for all $i$, i.e., $B^*$ is the dual basis to $B$.  However, we're not always that lucky.)


Notation:
\begin{itemize}
\item $\Nn$ = nonnegative integers;
\item For $u\in\Nn^k$, a \textbf{vector partition} $\lambda\partn u$ is an unordered sequence of vectors (\textbf{parts}) summing to $u$; zero vectors can be ignored.
\item Weight of a vector partition = sum of all entries in all vectors in it (ex.: $\wt(\{120,100,100,013\})=9$)
\item $\Mac_\uu$ = MacMahon symmetric functions of homogeneous multidegree $u$
\item $\sign(\lambda)=(-1)^{\text{number of parts with even sum}}$
\item $m_\lambda,e_\lambda,h_\lambda,p_\lambda,f_\lambda$: monomial, elementary, homogeneous, power-sum, and forgotten MacMahon symmetric functions (Rosas 327--328)
\item For a vector partition $\lambda$ in which each part $v_i$ occurs with multiplicity $m_i$, define
\[|\lambda|=\prod_im_i! \qqandqq \lambda!=\prod_i\prod_{x\in v_i}(x!)^{m_i}.\]
\end{itemize}

A vector partition is \textbf{unitary} if it is a partition of the all-ones vector $\mathbf{1}\in\Nn^k$.  There is an obvious bijection between unitary vector partitions and set partitions of $[k]$.  There is a whole theory of MacMahon symmetric functions of unitary partitions, due to Doubilet.  These are the graded pieces $\Mac_{(1)^k}$.

The \textbf{type} of a set partition $\pi=B_1|\cdots|B_\ell$, with respect to a vector $u=(u_1,\dots,u_r)$ of weight $n$, is the vector partition $\type_\uu(\pi)=\lambda=\lambda_1\cdots\lambda_\ell$ where $\lambda_k$ is the vector in $\Nn^r$ whose $i$th coordinate is
\[\#\{j\in B_k \mid u_1+\cdots+u_{i-1} < j \leq i_1+\cdots+u_i\}.\]
 the number of elements of $B_k$ such that
in the $i$th equivalence class.

For a vector $u$ of weight $n$ and a vector partition $\lambda$ of weight $n$, the number of set partitions $\pi$ of type $\type_\uu(\pi)$ is
\[\binom{u}{\lambda} := \frac{u!}{\lambda!\,|\lambda|}.\]

Define a scalar product on $\Mac$:
\begin{equation}\label{scalar-product}
\boxed{\scp{h_\lambda}{m_\mu}=\delta_{\lambda\mu}.}
\end{equation}
The idea is to use this for basis transitions, along the lines discussed earlier.  Specifically, formula~\eqref{expand-in-basis} says that for any $F\in\Mac_\uu$, we have
\begin{equation} \label{expand-Mac}
F=\sum_{\lambda\partn u} \langle h_\lambda,\,F\rangle m_\lambda \qqandqq
F=\sum_{\lambda\partn u} \langle m_\lambda,\,F\rangle h_\lambda.
\end{equation}

Thus we have to know how to compute inner products.

There is a ``lifting map'' $\hat\rho$ [Rosas 2001, Defn.~6] sending MacMahon functions to unitary (Doubilet) symmetric functions.  Specifically:
\begin{subequations}
\begin{align}
\binom{\uu}{\lambda}|\lambda|\,m_\lambda &\overset{\hat\rho}{\longmapsto} \sum_{\pi\colon \type_\uu(\pi)=\lambda} m_\pi \label{liftM}\\
\binom{\uu}{\lambda}\lambda!\,e_\lambda &\overset{\hat\rho}{\longmapsto} \sum_{\pi\colon \type_\uu(\pi)=\lambda} e_\pi\label{liftE}\\
\binom{\uu}{\lambda}\lambda!\,h_\lambda &\overset{\hat\rho}{\longmapsto} \sum_{\pi\colon \type_\uu(\pi)=\lambda} h_\pi\label{liftH}\\
\binom{\uu}{\lambda}\,p_\lambda &\overset{\hat\rho}{\longmapsto} \sum_{\pi\colon \type_\uu(\pi)=\lambda} p_\pi\label{liftP}\\
\binom{\uu}{\lambda}|\lambda|\,f_\lambda &\overset{\hat\rho}{\longmapsto} \sum_{\pi\colon \type_\uu(\pi)=\lambda} f_\pi\label{liftF}
\end{align}
\end{subequations}

Moreover, for all $f,g\in\Mac_\uu$ we have
\begin{equation} \label{Rosas-scalar}
\scp{f}{g}=\uu!\scp{\hat\rho(f)}{\hat\rho(g)}
\end{equation}
[Rosas, Prop.~7].

%  For instance, Doubilet's Formula \#1 says that for set partitions $\pi,\sigma$ of $[n]$ we have
%\begin{equation} \label{doubilet:1}
%\langle m_\pi,\,h_\sigma\rangle=n!\delta_{\pi\sigma}
%\end{equation}
%Therefore, if $\lambda$ and $\mu$ are vector partitions of $u$ of weight $n$, then
%\begin{align*}
%\langle m_\lambda,\,h_\mu\rangle
%&= u!\scp{\hat\rho(m_\lambda)}{\hat\rho(h_\mu)} && \text{(by~\eqref{Rosas-scalar})}\\
%&= u! \frac{1}{\binom{u}{\lambda}|\lambda|\binom{u}{\mu}\mu!} \scp{\sum_{\pi\colon \type_\uu(\pi)=\lambda} m_\pi}{\sum_{\sigma\colon \type_\uu(\sigma)=\mu} h_\sigma)} && \text{(by~\eqref{liftM} and~\eqref{liftH})}\\
%&= \delta_{\lambda\mu} \frac{u!\,n!\,\#\{\pi\colon \type_\uu(\pi)=\lambda\}}{\binom{u}{\lambda}^2|\lambda|\,\lambda!} && \text{(by linearity and~\eqref{doubilet:1})}
%\end{align*}
%
%Since Doubilet (Appendix 2) calculated the scalar products for all five families of unitary symmetric functions, we can lift them to MacMahon functions.  (We will need the lattice operations and  M\"obius and zeta functions in the set partition lattice $\Pi_n$

%Examples:
%P=posets.SetPartitions(4)
%x=SetPartition([[1,2,4],[3]])
%y=SetPartition([[1,2],[3,4]])
%hat0=P.minimal_elements()[0]
%hat1=P.maximal_elements()[0]
%P.moebius_function(hat0,hat1)
%P.join(x,y)
%P.meet(x,y)


Prop.~\ref{expansion} together with the scalar product~\eqref{scalar-product} says in particular that for $F\in\Mac_\uu$ we have
\begin{equation} \label{expand-HM}
F = \sum_{\lambda\partn\uu} \langle m_\lambda,\,F\rangle\, h_\lambda,
\qquad\qquad
F = \sum_{\lambda\partn\uu} \langle h_\lambda,\,F\rangle\, m_\lambda.
\end{equation}
%In particular (I am going to be very repetitive here),
%\begin{align}
%m_\mu &= \sum_{\lambda\partn\uu} \langle m_\lambda,\,m_\mu\rangle\, h_\lambda, &
%m_\mu &= \sum_{\lambda\partn\uu} \langle h_\lambda,\,m_\mu\rangle\, m_\lambda, \\
%h_\mu &= \sum_{\lambda\partn\uu} \langle m_\lambda,\,h_\mu\rangle\, h_\lambda, &
%h_\mu &= \sum_{\lambda\partn\uu} \langle h_\lambda,\,h_\mu\rangle\, m_\lambda, \\
%e_\mu &= \sum_{\lambda\partn\uu} \langle m_\lambda,\,e_\mu\rangle\, h_\lambda, &
%e_\mu &= \sum_{\lambda\partn\uu} \langle h_\lambda,\,e_\mu\rangle\, m_\lambda, \\
%p_\mu &= \sum_{\lambda\partn\uu} \langle m_\lambda,\,p_\mu\rangle\, h_\lambda, &
%p_\mu &= \sum_{\lambda\partn\uu} \langle h_\lambda,\,p_\mu\rangle\, m_\lambda, \\
%f_\mu &= \sum_{\lambda\partn\uu} \langle m_\lambda,\,f_\mu\rangle\, h_\lambda, &
%f_\mu &= \sum_{\lambda\partn\uu} \langle h_\lambda,\,f_\mu\rangle\, m_\lambda.
%\end{align}
Doubilet (Appendix 2) calculated the scalar products for all five families of unitary symmetric functions (which requires things like lattice operations and  M\"obius and zeta functions in the set partition lattice $\Pi_n$).  Therefore, we should be able to compute scalar products, and thus basis expansions, for all MacMahon functions.
For instance, for $\mu\partn\uu$, we can expand a power-sum MacMahon function $p_\mu$ in the monomial basis as follows:
\begin{align*}
p_\mu &= \sum_{\lambda\partn\uu} \langle h_\lambda,\,p_\mu\rangle\, m_\lambda
	&&\text{(by~\eqref{expand-Mac})}\\
&=  \uu! \sum_{\lambda\partn\uu} \langle \hat\rho(h_\lambda),\,\hat\rho(p_\mu)\rangle\, m_\lambda
	&&\text{(by~\eqref{Rosas-scalar})}\\
&=  \uu! \sum_{\lambda\partn\uu} \left\langle \frac{1}{\binom{\uu}{\lambda}\lambda!} \sum_{\pi\colon \type_\uu(\pi)=\lambda} h_\pi, \ \ \frac{1}{\binom{\uu}{\mu}} \sum_{\sigma\colon \type_\uu(\sigma)=\mu} p_\sigma \right\rangle\, m_\lambda
	&&\text{(by~\eqref{liftH} and~\eqref{liftP})}\\
&=  \frac{\uu!}{\binom{\uu}{\mu}} \sum_{\lambda\partn\uu} \frac{1}{\binom{\uu}{\lambda}\lambda!}
    \sum_{\substack{\pi\colon \type_\uu(\pi)=\lambda\\ \sigma\colon \type_\uu(\sigma)=\mu}}
    \left\langle h_\pi,\,p_\sigma \right\rangle\, m_\lambda
	&&\text{(by bilinearity)}\\
&=  \frac{\uu!}{\binom{\uu}{\mu}} \sum_{\lambda\partn\uu} \frac{1}{\binom{\uu}{\lambda}\lambda!}
    \sum_{\substack{\pi\colon \type_\uu(\pi)=\lambda\\ \sigma\colon \type_\uu(\sigma)=\mu}}
    n! \zeta(\sigma,\pi) m_\lambda
	&&\text{(by~Doubilet formula \#3)}\\
&=  \frac{n!\,\uu!}{\binom{\uu}{\mu}} \sum_{\lambda\partn\uu} \frac{1}{\binom{\uu}{\lambda}\lambda!}
    \sum_{\substack{\pi\colon \type_\uu(\pi)=\lambda\\ \sigma\colon \type_\uu(\sigma)=\mu\\ \sigma\leq\pi}}
    m_\lambda
\end{align*}
which is something we could implement.


% Doubilet's Formula \#1 says that for set partitions $\pi,\sigma$ of $[n]$ we have
%\begin{equation} \label{doubilet:1}
%\langle m_\pi,\,h_\sigma\rangle=n!\delta_{\pi\sigma}
%\end{equation}
%Therefore, if $\lambda$ and $\mu$ are vector partitions of $u$ of weight $n$, then
%\begin{align*}
%\langle m_\lambda,\,h_\mu\rangle
%&= u!\scp{\hat\rho(m_\lambda)}{\hat\rho(h_\mu)} && \text{(by~\eqref{Rosas-scalar})}\\
%&= u! \frac{1}{\binom{u}{\lambda}|\lambda|\binom{u}{\mu}\mu!} \scp{\sum_{\pi\colon \type_\uu(\pi)=\lambda} m_\pi}{\sum_{\sigma\colon \type_\uu(\sigma)=\mu} h_\sigma)} && \text{(by~\eqref{liftM} and~\eqref{liftH})}\\
%&= \delta_{\lambda\mu} \frac{u!\,n!\,\#\{\pi\colon \type_\uu(\pi)=\lambda\}}{\binom{u}{\lambda}^2|\lambda|\,\lambda!} && \text{(by linearity and~\eqref{doubilet:1})}
%\end{align*}

\end{document}